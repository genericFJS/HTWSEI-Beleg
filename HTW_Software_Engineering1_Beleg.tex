\newcommand{\customDir}{}
\input{\customDir _LaTeX_master/LaTeX_master_setup.sty}

%\setboolean{twosided}{true}
%\setCustomDocumentClass{scrartcl}
%\setCustomDesign{htw}
\setCustomSlidePath{Vorlesung/VO}
\setCustomSlideScale{1.5}

\setCustomTitle{Software Engineering 1}
\setCustomSubtitle{Beleg}
\setCustomAuthor{Falk-Jonatan Strube}
%\setCustomNoteA{TitlepageNoteBeforeAuthor}
\setCustomNoteB{Vorlesung von Prof. Dr. Hauptmann}

\setCustomSignature{\footnotesize{\textcolor{darkgray}{von Luga, Pour, \\Retsch, Strube}}}	% Formatierung der Signatur in der Fußzeile
\setCustomTitleAuthor{\textcolor{darkgray}{\customAuthor}}	% Formatierung des Autors auf dem Titelblatt

\input{\customDir _LaTeX_master/LaTeX_master.sty}
\input{\customDir _LaTeX_master/LaTeX_master_macros.sty}

\usepackage{adjustbox}
%\bibliography{\customDir _Literatur/HTW_Literatur.bib}

\begin{document}

%\selectlanguage{english}
\maketitle
\newpage
\tableofcontents
\newpage

\chapter*{Aufgabe 1: Anforderungsanalyse}
\addtocounter{chapter}{1}
\addcontentsline{toc}{chapter}{Aufgabe 1: Anforderungsanalyse}
\section{Funktionale Anforderungen}
\subsection*{Anwesenheit verwalten}
\begin{itemize}
\item Anwesenheit erfassen
\begin{itemize}
\item eintragen
\item austragen 
\item zurückweisen
\end{itemize}
\item Übersicht der Anwesenden bereitstellen
\begin{itemize}
\item für Sicherheitsdienst von allen
\item für Abteilungsleiter von Abteilung
\end{itemize}
\end{itemize}
\subsection*{Abwesenheit verwalten}
\begin{itemize}
\item Urlaub verwalten
\begin{itemize}
\item Anträge verwalten
\begin{itemize}
\item Einreichen
\item Stornieren
\item Genehmigen
\item Ablehnen
\end{itemize}
\item Vorschläge verwalten
\begin{itemize}
\item Vorschlag erstellen
\item Vorschlag annehmen
\item Vorschlag ablehnen
\end{itemize}
\item Urlaubsinformationen anzeigen %(Mitarbeiter/Abteilungsleiter)
\end{itemize}
\item Krankheitszeiten verwalten
\begin{itemize}
\item Krankheitsdaten erfassen 
\begin{itemize}
\item Urlaubsdaten anpassen
\end{itemize}
\end{itemize}
\end{itemize}
\subsection*{Auswertung}
\begin{itemize}
\item Auswertung der Arbeitszeit
\begin{itemize}
\item erstellen
\item versenden
\end{itemize}
\item Auswertung der Gesamtbilanz
\begin{itemize}
%\item erstellen
\item anfordern
\end{itemize}
\item Auswertung der Urlaubszeit
\begin{itemize}
%\item erstellen
\item anfordern
\end{itemize}
\end{itemize}

\section{Qualitätsanforderungen}

\begin{itemize}
\item hat Benutzerschnittstelle
\item läuft selbstständig
\item ist unterbrechungsfrei nutzbar %
\item ist konsistent
\end{itemize}

\section{Rahmenbedingungen}
\subsection{technisch/technologisch}
\begin{itemize}
\item Hardware für Benutzerinteraktion wird bereitgestellt
\item Mitarbeiterausweis ist vorhanden
\item Mitarbeiterausweis von anwesenden MA wird auf Firmengelände Raumweise ermittelt
\end{itemize}
\subsection{rechtlich}
\begin{itemize}
\item Mitarbeiterdaten werden gespeichert
\end{itemize}
\subsection{organisatorisch}
\begin{itemize}
\item Arbeitstag = 8 Stunden
\item Arbeitswoche = 40 Stunden
\item Zugriff auf Jahreskalender
\end{itemize}

\chapter*{Aufgabe 2: Auslöser-Reaktions-Tabelle}
\setcounter{section}{0}
\addtocounter{chapter}{1}
\addcontentsline{toc}{chapter}{Aufgabe 2: Auslöser-Reaktions-Tabelle}

\adjustbox{width=\linewidth}{
\begin{tabular}{L{.25} | L{.25} | L{.2} | L{.2} | L{.4}}
Ereignis & funktionale Anforderungen & Eingabe-Daten & Ausgabe-Daten &Bemerkung (Q,R)\\
\hline\hline
Person betritt/verlässt Firma & Anwesenheit erfassen & Mitarbeiterausweis & Ablehnung & Ablehnung nur bei ungültiger MA\_ID\\\hline
Mitarbeiter (MA) beantragt Urlaub & Urlaubsantrag einreichen & Urlaubsantrag &  & \\\hline
MA ruft Urlaubsinform. ab& Urlaubsinformationen anzeigen & MA\_ID & Urlaubsinformationen & Abteilungsleiter kann zusätzlich Urlaubsinf. seiner MA abrufen\\\hline
MA nimmt Urlaubsvorschlag an & Urlaubsvorschlag annehmen & Urlaubs\_ID & & \\\hline
MA lehnt Urlaubsvorschlag ab & Urlaubsvorschlag ablehnen & Urlaubs\_ID & & \\\hline
MA storniert Urlaub & Urlaub stornieren & Urlaubs\_ID & & Urlaubsinformation werden gelöscht, wenn Voraussetzung erfüllt(offen / abgelehnt / genehmigt und nicht angetreten)\\\hline
MA meldet sich krank & Krankmeldung erfassen & Krankenschein & & Urlaubsdaten werden korrigiert \\\hline
Abteilungsleiter genehmigt Urlaubsantrag & Urlaubsantrag genehmigen & Urlaubs\_ID &  &\\\hline
Abteilungsleiter lehnt Urlaubsantrag ab& Urlaubsantrag ablehnen & Urlaubs\_ID &  & \\\hline
Abteilungsleiter unterbreitet Urlaubsvorschlag an MA& Urlaubsvorschlag erstellen& MA\_ID & & \\\hline
Abteilungsleiter fordert Auswertung an& Auswertung anfordern & 2* Zeitraum & Auswertung & Zeiträume: in Zukunft für Gesamtbilanz, in Vergangenheit für Urlaubszeitbilanz. Auswertung besteht aus Gesamtbilanz, Urlaubsbilanz und Anwesenheitsliste der Abteilung des Abteilungsleiters\\\hline
Neue Stunde zwischen 22 und 6 Uhr & Anwesenheitsliste bereitstellen & & Anwesenheitsliste & aller Mitarbeiter an Sicherheitsdienst\\ \hline
Woche ist zu Ende& Arbeitszeitauswertung versenden & & & E-Mail versenden\\
\end{tabular}
}
\chapter*{Aufgabe 3: Datenstruktur}
\setcounter{section}{0}
\addtocounter{chapter}{1}
\addcontentsline{toc}{chapter}{Aufgabe 3: Datenstruktur}
\section{Mitarbeiterausweis}
\begin{itemize}
\item MA\_ID
\item Gültigkeit
\item Vorname
\item Name
\item Geburtsdatum
\end{itemize}
\section{Ablehnung}
\begin{itemize}
\item Ablehnung
\end{itemize}
\section{Anwesenheitsliste}
Liste: 
\begin{itemize}
\item MA\_ID
\item Name
\item Raum
\end{itemize}
\section{Urlaubsantrag}
\begin{itemize}
\item MA\_ID
\item Urlaubs\_ID
\item Zeitraum
\end{itemize}
\section{Urlaubsinformationen}
\begin{itemize}
\item Anzahl verbrauchter Urlaubstage
\item Anzahl verbleibender Urlaubstage
\end{itemize}
Liste: 
\begin{itemize}
\item Urlaubs\_ID
\item Zeitraum
\item Status
\end{itemize}
\section{Krankenschein}
\begin{itemize}
\item Mit\_ID
\item Zeitraum
\end{itemize}
\section{Zeitraum}
\begin{itemize}
\item Start-Datum
\item End-Datum
\end{itemize}
\section{Auswertung}
\begin{itemize}
\item Gesamtbilanz
\begin{itemize}
\item Sollarbeitszeit
\item tatsächliche Arbeitsstunden
\item der Urlaubstage
\item Krankheitstage
\item Überstunden
\end{itemize}
jeweils: 
\begin{itemize}
\item[]
\begin{itemize}
\item absolut
\item prozentual
\end{itemize}
\end{itemize}
\item Urlaubszeitbilanz
\begin{itemize}
\item beantragte Urlaubstage\\
(bezogen auf Gesamtarbeitszeit)
\begin{itemize}
\item absolut
\item prozentual
\end{itemize}
\end{itemize}
\item Anwesenheitsliste\\
(der Mitarbeiter in der Abteilung)
\end{itemize}
%\newpage

\chapter*{Aufgabe 4: Kontextdiagramm}
\setcounter{section}{0}
\addtocounter{chapter}{1}
\addcontentsline{toc}{chapter}{Aufgabe 4: Kontextdiagramm}

%\paragraph{Akteure}
%\begin{itemize}
%	\item Mitarbeiter
%	\item Wachdienst
%	\item Abteilungsleiter
%	\item System
%	\item Sachbearbeiterin der Personalabteilung
%\end{itemize}

\begin{center}
	\includegraphics[scale=0.4]{Kontextdiagramm.PNG}
\end{center}

\chapter*{Aufgabe 5: Grobes Anwendungsfalldiagramm}
\setcounter{section}{0}
\addtocounter{chapter}{1}
\addcontentsline{toc}{chapter}{Aufgabe 5: Grobes Anwendungsfalldiagramm}

\begin{center}
	\includegraphics[scale=0.4]{GrobesAnwendungsfalldiagramm.PNG}
\end{center}


\chapter*{Aufgabe 6: Detailliertes Anwendungsfalldiagramm von Urlaub planen}
\setcounter{section}{0}
\addtocounter{chapter}{1}
\addcontentsline{toc}{chapter}{Aufgabe 6: Detailliertes Anwendungsfalldiagramm von Urlaub planen}

\begin{center}
	\includegraphics[scale=0.4]{UrlaubPlanen_AWF.PNG}
\end{center}

\chapter*{Aufgabe 7:}
\setcounter{section}{0}
\addtocounter{chapter}{1}
\addcontentsline{toc}{chapter}{Aufgabe 7: Anwendungsfallbeschreibung}

\section{Jonatan: Urlaub einreichen}

\subsection{Textliche Beschreibung}

\subsubsection{Kurzbeschreibung}

Ein Mitarbeiter reicht einen Urlaubsantrag ein.

\subsubsection{Akteur}

Mitarbeiter

\subsubsection{Vorbedingungen}

\begin{itemize}
\item Urlaubsantrag
\end{itemize}

Urlaubsantrag =
\begin{itemize}
\item[] MA\_ID
\item[+] Urlaubs\_ID
\item[+] Zeitraum
\end{itemize}

Zeitraum = 
\begin{itemize}
\item[] Start-Datum
\item[+] End-Datum
\end{itemize}

\subsubsection{Nachbedingungen}
\begin{itemize}
\item Urlaubsantrag ist im System gespeichert
\end{itemize}

\subsubsection{Trigger}
\begin{itemize}
\item Urlaubsantrag stellen
\end{itemize}

\subsubsection{Szenarios}

\paragraph{Hauptszenario:} Mitarbeiter gibt validen Urlaubsantrag ein.
\begin{enumerate}
\item Mitarbeiter möchte Urlaubsantrag stellen
\item Mitarbeiter gibt Urlaubseintrag in System ein
\item System überprüft Urlaubsantrag
\item System trägt Urlaubsantrag in Urlaubsinformationen des Mitarbeiters ein
\end{enumerate}

\paragraph{Alternativszenario:} Mitarbeiter gibt keinen validen Urlaubsantrag ein.
\begin{enumerate}
\item Mitarbeiter möchte Urlaubsantrag stellen
\item Mitarbeiter gibt Urlaubseintrag in System ein \label{jo-alt-back}
\item System überprüft Urlaubsantrag
\item Zurück zu \ref{jo-alt-back}.
\end{enumerate}

\subsubsection{Weiterführende Informationen}
keine

\subsection{Aktivitätsdiagramm}

\begin{center}
	\includegraphics[width=0.9\linewidth]{Urlaub_einreichen.pdf}
\end{center}

\subsection{Satzschablonen}

\section{Simon: Urlaubsvorschlag annehmen}


\section{Ragnar: Urlaubsvorschlag ablehnen}

%\printbibliography
 
\end{document}